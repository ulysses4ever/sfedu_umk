\input umk_preamble

\input general_info.tex

\begin{document}

%\tableofcontents

\section{Цели и задачи освоения дисциплины}
% цель и задачи!

\section{Место дисциплины в структуре ООП ВПО}

% использовать \ssect[Абв] или \ssect для нумерации подразделов 
% с факультативным заголовком

	\ssect Учебная дисциплина \thecourse{}
(\theyearofstudy~курс, \theterm~семестр) относится к \ulinepad{
математическому и естественнонаучному
% / 
% профессиональному
% обычно видно по учебному плану: 
% м. и ес. там обозначен Б2, п. -- Б3 
% учебные планы ЮФУ: http://sfedu.ru/www/view_plans.startup
} циклу.

	\ssect % пререквизиты, например:
% Для изучения курса \thecourse (\theyearofstudy~курс, \theterm~семестр)
% студенту достаточно владеть >>>???основами программирования>>>, 
% полученными в
% базовом курсе «Компьютерные науки», 
% который читается в 1–2 семестре 1 курса.

	\ssect % курс является пререквизитом для... например:
% В дальнейшем материал данного курса будет использоваться в ряде курсов,
% изучаемых на 2–4 курсах и предполагающих использование компьютерных
% технологий, в том числе: ???

\section{Требования к результатам освоения содержания дисциплины}

	\ssect
Процесс изучения дисциплины направлен на формирование элементов следующих компетенций в соответствии с ФГОС ВПО (ОС ЮФУ) и ООП ВПО по данному направлению подготовки:
% общий перечень компетенций по направлению подготовки обычно 
% можно найти в госстандарте или в ООП вуза. ФГОСы размещены на одном из 
% специализированных сайтов Минобра. Сейчас это:
% http://fgosvo.ru/fgosvpo/7/6/1/28
% ООП направлений ЮФУ должны быть на офсайте
% на данный момент (середина 2014) это:
% http://sfedu.ru/www/edu.NaprPodg_show?v_snp_date_in=01.01.2009
% существующий на данный момент перечень для ПМИ и ФИИТ вынесен в файл:
% https://docs.google.com/document/d/12WMvvjyVEkF9S5gQVCI26zMnVXGQJcVEv18Qjz_x9yk/edit?usp=sharing
\begin{enumerate}
\rusitems % нумерация кириллическими буквами
	\item общекультурных (ОК):
	\begin{itemize}
		\item ???
	\end{itemize}

	\item профессиональных (ПК):
	\begin{itemize}
		\item ???
	\end{itemize}
\end{enumerate}

В результате освоения дисциплины обучающийся должен:

\textbf{Знать:}
	\begin{itemize}
		\item ???
	\end{itemize}

\textbf{Уметь:}
	\begin{itemize}
		\item ???
	\end{itemize}

\textbf{Владеть:}
	\begin{itemize}
		\item ???
	\end{itemize}

\section{Содержание и структура дисциплины}
	
	\ssect[Содержание модулей дисциплины]

% общая информация о модулях должна быть представлена в данном файле:
\input my_units.tex

% для изложения содержания каждого модуля используется команда myunit,
%   она создаёт бокс с рамкой, это отличается от табличного формата,
%   данного в образце УМК/РПД, однако их формат:
%   1) неразумно использует пространство страницы (пустые колонки)
%   2) сложно или невозможно воспроизвести в LaTeX, 
%		см. http://tex.stackexchange.com/q/192728/7460

%	имена модулей должны быть заданы в файле my_units.tex
%		здесь указывается только содержание и формы контроля

%	Главный корпус требует от 2 до 4 модулей (включительно)


% Модуль 1
\myunit
	{Современные многоуровневые машины. Понятия архитектуры и организации компьютера. Развитие вычислительной техники. Разнообразие компьютеров: от «одноразовых компьютеров» до суперкомпьютеров, закон Мура. Обзор центрального процессора. Логическая эквивалентность аппаратного и программного обеспечения, принцип микропрограммного управления. Проектирование современных процессоров, CISC и RISC. Параллельные вычислительные системы. 
	Оперативная память, вопросы организации ОЗУ: ячейки, порядок байт. Сверхоперативная память: кэш-память. 
	Иерархия памяти. Вторичная память, устройство накопителей на магнитных дисках, геометрия CHS, основные интерфейсы. Помехоустойчивое кодирование. 
	Подсистема ввода-вывода, системные шины, чипсет.}
	{Домашние работы в виде тестов. Контрольная работа в виде теста.}

% Модуль 2
\myunit
	{Цифровой логический уровень. Транзисторы, вентили, логические схемы, печатные платы. Комбинационные схемы, секвенциональные схемы (схемы памяти), тактовые генераторы. Вопросы проектирования системных шин: ширина, перекосы, мультиплексирование, арбитраж, (а)синхронность. Последовательные и параллельные шины.
	Уровень микроархитектуры. Тракт данных, внутренние регистры, организация управляющего устройства в виде микропрограммы. Пример микроархитектуры Mic-1 и реализация простейших инструкций IJVM с её помощью.
	Вопросы проектирования уровня набора команд. Представление числовых типов данных: знаковые целые числа, числа с плавающей точкой (стандарт IEEE~754).}
	{Домашние работы в виде тестов. Контрольная работа в виде теста.}

	\ssect[Структура дисциплины]
% Вступительная фраза и две таблицы с расчасовками:
\printhours

	\ssect[Лабораторные работы]

%\mylab{Имя лабы}{часы}
\printlabs{%
	\mylab{Простейшие программы, арифметика и циклы LOOP}{4}%
	\mylab{Массивы. Условные и безусловные переходы}{4}%
	\mylab{Интерфейс системных вызовов. Простейшие подпрограммы}{4}%
	\mylab{Подпрограммы (продолжение)}{4}%
	\mylab{Контрольная лабораторная работа}{2}%
	%
	\incMyunit
	%
	\mylab{Цепочечные инструкции}{4}%
	\mylab{Работа с файлами}{4}%
	\mylab{Микропрограммирование}{4}%
	\mylab{Контрольная лабораторная работа}{2}%
}

\section{Образовательные технологии}



\section{Оценочные средства для текущего контроля успеваемости и промежуточной аттестации}

\section{Учебно-методическое обеспечение дисциплины}

	\ssect[Основная литература]\label{main-lit}

\begin{enumerate}
	\item Tanenbaum A., Ostin T. Structured Computer Organization / Prentice Hall; 6 edition (August 4, 2012). 800 p.\\
	Перевод: Таненбаум Э., Остин Т. Архитектура компьютера / 6-е изд.(+CD) — СПб.: Питер, 2013. — 816 с.
\end{enumerate}

	\ssect[Дополнительная литература]
\begin{enumerate}
	\item Stallings W. Computer Organization and Architecture /  Prentice Hall; 9 edition (March 11, 2012). 792 p.\\
	Перевод: Столлингс У. Структурная организация и архитектура компьютерных систем / М: Вильямс, 2002. 896 с.
\end{enumerate}

	\ssect[Список авторских методических разработок]
	\label{author-res}
\begin{enumerate}
	\item А.\,М.~Пеленицын, Н.\,Н.~Ячменёва. Методические указания к практикуму по курсу «Архитектура компьютера» [Электронный ресурс]\\
	\url{http://open-edu.sfedu.ru/node/2622}
\end{enumerate}

	\ssect[Интернет-ресурсы]\label{online-res}
\begin{enumerate}
	\item Сопроводительные материалы к учебнику [1] п.~\ref{main-lit}:
	\begin{otherlanguage}{english}
	Structured Computer Organization, 6/E 
	\end{otherlanguage}
	[Электронный ресурс]\\
	\url{http://www.pearsonhighered.com/educator/product/Structured-Computer-Organization-6E/9780132916523.page}

	\item Википедия: Свободная энциклопедия, английский раздел [Электронный ресурс]\\
	\url{http://en.wikipedia.org/wiki/Main_Page}
\end{enumerate}

\section{Материально-техническое обеспечение дисциплины}
	
	\ssect[Учебно-лабораторное оборудование]
Лекции проводятся в мультимедийном классе с презентационным оборудованием (проектором и экраном либо интерактивной доской). Лабораторные занятия проводятся в дисплейных классах с персональными компьютерами по числу, не уступающему числу студентов.

	\ssect[Программные средства]

Компьютеры в дисплейных классах должны быть снабжены операцинной системой GNU/Linux, желательно в виде дружественного к пользователю дистрибутива (например, Ubuntu Linux).

На компьютерах в дисплейном классе должны быть распакованы в каталог \texttt{/bin} или \texttt{\textasciitilde/bin} три утилиты ассемблирования (as88/t88/s88) из сопроводительных материалов к учебнику Таненбаума (см. [1] в п.~~\ref{online-res}). Для выполнения лабораторной работы по микропрограммированию необходимо наличие каталога Mic1MMV (с содержимым) оттуда же. Для его работы требуется виртуальная машина Java версии не ниже 1.4.

Для редактирования кода рекомендуется иметь установленным специализированный редактор с подсветкой синтаксиса языка ассемблера, например Geany или jEdit (оба находятся в частности в репозиториях Ubuntu Linux).

В отдельных случаях возможна работа на комптьютерах под управлением операционной системы семества Windows, однако здесь требуется дополнительная настройка утилит ассмеблирования, описанная в методических указаниях~[1] п.~\ref{author-res}.


	\ssect[Технические и электронные средства]

Учёт уктивности студентов на курсе и основные материалы размещены в системе Moodle, равёрнутой в сети университета по адресу \url{http://edu.mmcs.sfedu.ru}. В дисплейных классах требуется доступ к этому ресурсу посредством бразуера актуальной версии.

\section{Учебная карта дисциплины}

\input ukd_body.tex

\end{document}